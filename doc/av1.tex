\documentclass[17pt]{extarticle}

\usepackage[utf8x]{inputenc}
\usepackage{listings}
\usepackage{color}
\usepackage{graphicx}
\usepackage{float}
\graphicspath{{graphs/}}
\definecolor{dkgreen}{rgb}{0,0.6,0}
\definecolor{gray}{rgb}{0.5,0.5,0.5}
\definecolor{mauve}{rgb}{0.58,0,0.82}

\lstset{
    frame=tb,
    language=R,
    aboveskip=3mm,
    belowskip=3mm,
    showstringspaces=false,
    columns=flexible,
    basicstyle={\small\ttfamily},
    numbers=none,
    numberstyle=\tiny\color{gray},
    keywordstyle=\color{blue},
    commentstyle=\color{dkgreen},
    stringstyle=\color{mauve},
    breaklines=true,
    breakatwhitespace=true,
    tabsize=4,
    basicstyle=\tiny
}

\title{Estudo descritivo sobre os dados do Sistema de Avaliação da Educação Básica (Saeb) utilizando a linguagem R}
\author{Rafael da Silva Albuquerque}
\date{03.2024}

\begin{document}

\begin{figure}[t]
    \centering
    \includegraphics[width=0.5\linewidth]{doc/unifor-logo.png}
    \label{fig:my_label}
\end{figure}
\maketitle

\newpage
\section{Introdução}
dgmaipmgipa

\newpage
\section{Objetivo Específico}
$\rightarrow$ Fazer uma tabela com as frequências simples e relativas para as Variáveis: já foi reprovado, já abandonou a escola; gosta de estudar Língua Portuguesa; gosta de estudar Matemática; faz o dever de casa de Matemática.\newline
$\rightarrow$ Montar tabela cruzada com as frequências simples e relativas para as variáveis: já foi reprovado versus já abandonou a escola; gosta de estudar Matemática versus faz o dever de casa de Matemática.\newline
$\rightarrow$ Montar um gráfico de barra para a variável quando entrou na escola;\newline
$\rightarrow$ Montar um gráfico de setor para as variáveis: gosta de estudar Matemática; gosta de estudar Língua Portuguesa.\newline
$\rightarrow$ Determinar as medidas de posição e separatrizes para as variáveis: Nota Português; Nota Matemática.\newline
$\rightarrow$ Determinar as medidas de dispersão para as variáveis: Nota Português; Nota Matemática.\newline
$\rightarrow$ Determinar o boxplot para as variáveis: Nota Português; Nota Matemática.\newline
$\rightarrow$ Determinar o histograma para as variáveis: Nota Português; Nota Matemática.\newline

\section{Análise e Interpretação dos dados}
\subsection{Gráfico de barra para a variável "Quando você entrou na escola?"}
\begin{figure}[H]
    \includegraphics[width=0.7\linewidth]{quando_entrou_na_escola.png}
    \centering
\end{figure}
Este gráfico mostra que a maioria dos alunos do ensino básico entrou na escola na pré-escola, enquanto a minoria entrou depois da primeira série. O segundo maior grupo entrou na escola na creche, e o segundo menor grupo entrou na primeria série ou primeiro ano. \newline
Código fonte: \newline
\begin{lstlisting}
src <- read.table(file="sources\\DadosSaeb.csv", sep=";", header=TRUE)

quando_entrou <- table(src$"Quando.voce.entrou.na.escola.")
png(".\\graphs\\quando_entrou_na_escola.png")
relatorio3 <- barplot(
    height=as.numeric(quando_entrou),
    names=c(),
    col=c("#003f5c", "#58508d", "#bc5090", "#ff6361"),
    ylim=c(0, 60000),
    main="Quando voce entrou na escola?",
)
par(xpd=TRUE)
legend(
    "topleft",
    inset=c(0, 0),
    legend=c("Depois da primeira serie", "Na creche (0 a 3 anos)", "Na pre-escola (4 a 5 anos)", "Na primeira serie ou primeiro ano (6 a 7 anos)"),
    col=c("#003f5c", "#58508d", "#bc5090", "#ff6361"),
    lty=1,
    cex=1,
    title="Respostas",
    text.font=4,
    bg="white"
)
dev.off()
\end{lstlisting}

\subsection{Gráfico de setor para a variável "Você gosta de estudar Matemática?"}
\begin{figure}[H]
    \includegraphics[width=0.7\linewidth]{gosta_mat.png}
    \centering
\end{figure}
Este gráfico mostra que a maioria dos estudantes do ensino básico afirmam gostar de estudar Matemática, ou seja, \(\frac{71771}{99999} \approx 0.717\) ou 71.7 a cada 100 estudantes. \newline
Código-fonte:
\begin{lstlisting}
src <- read.table(file="sources\\DadosSaeb.csv", sep=";", header=TRUE)

gosta_mat <- table(src$"Voce.gosta.de.estudar.Matematica.")

png(".\\graphs\\gosta_mat.png")
relatorio4 <- pie(
    as.numeric(gosta_mat),
    col=c("#ffa600", "#003f5c"),
    labels=as.numeric(gosta_mat),
    main="Voce gosta de estudar Matematica?"
)
legend(
    "topleft",
    legend=c("Sim", "Nao"),
    col=c("#003f5c", "#ffa600"),
    lty=1,
    cex=1,
    title="Respostas",
    text.font=4,
    bg="white"
)
dev.off()
\end{lstlisting}

\subsection{Gráfico de setor para a variável "Você gosta de estudar Língua Portuguesa?"}
\begin{figure}[H]
    \includegraphics[width=0.7\linewidth]{gosta_lp.png}
    \centering
\end{figure}
Este gráfico mostra que a maioria dos estudantes do ensino básico afirmam gostar de estudar Língua Portuguesa, ou seja, \(\frac{81626}{99999} \approx 0.816\) ou 81.6 a cada 100 estudantes. \newline
Código-fonte:
\begin{lstlisting}
src <- read.table(file="sources\\DadosSaeb.csv", sep=";", header=TRUE)

gosta_lp <- table(src$"Voce.gosta.de.estudar.Lingua.Portuguesa.")

png(".\\graphs\\gosta_lp.png")
relatorio5 <- pie(
    as.numeric(gosta_lp),
    col=c("#ffa600", "#003f5c"),
    labels=as.numeric(gosta_lp),
    main="Voce gosta de estudar Lingua Portuguesa?",
)
legend(
    "topleft",
    legend=c("Sim", "Nao"),
    col=c("#003f5c", "#ffa600"),
    lty=1,
    cex=1,
    title="Respostas",
    text.font=4,
    bg="white"
)
dev.off()
\end{lstlisting}

\subsection{Boxplot para a variável "Nota Matemática"}
\begin{figure}[H]
    \includegraphics[width=0.7\linewidth]{box_mat.png}
    \centering
\end{figure}

\subsection{Boxplot para a variável "Nota Português"}
\begin{figure}[H]
    \includegraphics[width=0.7\linewidth]{box_lp.png}
    \centering
\end{figure}

\subsection{Histograma para a variável "Nota Matemática"}
\begin{figure}[H]
    \includegraphics[width=0.7\linewidth]{hist_mat.png}
    \centering
\end{figure}

\subsection{Histograma para a variável "Nota Português"}
\begin{figure}[H]
    \includegraphics[width=0.7\linewidth]{hist_lp.png}
    \centering
\end{figure}

\newpage
\section{Referências bibliográficas}

\end{document}
